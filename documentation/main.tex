\documentclass{article}
\usepackage{graphicx} % Required for inserting images
\usepackage[T1]{fontenc}
\usepackage[utf8]{inputenc}
\usepackage[margin=1.5in]{geometry}
% \usepackage[polski]{babel}
\usepackage[polish]{babel}
\usepackage{amsmath}

\usepackage[
backend=biber,
style=numeric-comp,
sorting=ynt
]{biblatex}
\addbibresource{references.bib}


\title{Algorytmy Inspirowane Biologicznie – Optymalizacja Kolonii Mrówek}
\date{Kraków 2023}
\author{Krzysztof Ćwiertnia, Wojciech Wiśniewski}

\begin{document}

\maketitle

\section{Cel projektu}
Celem projektu jest rozwiązanie problemu komiwojażera poprzez wykorzystanie
algorytmu mrówkowego.

\section{Wstęp teoretyczny}
Optymalizacja kolonii mrówek (ACO) to meta heurystyka, w której kolonia
sztucznych mrówek współpracuje w znajdowaniu dobrych rozwiązań trudnych
problemów optymalizacji dyskretnej. Współpraca jest kluczowym elementem projektu
algorytmów ACO: Wybór polega na przydzieleniu zasobów obliczeniowych do zestawu
stosunkowo prostych agentów (sztucznych mrówek), które komunikują się pośrednio
przez stygmatyzację, czyli poprzez komunikację pośrednią za pośrednictwem
środowiska.

Algorytmy ACO mogą być wykorzystywane do rozwiązywania zarówno statycznych, jak
i dynamicznych kombinatorycznych problemów z optymalizacją. Problemy statyczne
to takie, w których cechy problemu są podane raz na zawsze, kiedy problem jest
zdefiniowany i nie ulegają zmianie, gdy problem jest rozwiązywany.
Paradygmatycznym przykładem takich problemów jest tzw. Problem Komiwojażera
(TSP), w którym lokalizacje miast i ich względne odległości są częścią definicji
problemu i nie zmieniają się w czasie wykonywania. Przeciwnie, dynamiczne
problemy definiuje się jako funkcję pewnych wielkości, których wartość wyznacza
dynamika systemu bazowego. Instancja problemu zmienia się zatem w czasie
wykonywania i algorytm optymalizacji musi być zdolny do adaptacji do
zmieniającego się środowiska. Przykładem takiej sytuacji, są problemy z
trasowaniem pracy, w których ruch danych i topologia sieci mogą się różnić w
czasie.\cite{MIT}

\section{Problem Komiwojażera}
Problem komiwojażera (TSP) jest dobrze znanym i szeroko badanym problemem z
dziedziny optymalizacji kombinatorycznej i przyciąga informatyków, matematyków i
innych. Jest znany jako klasyczny problem NP-zupełny, który ma wielkie
przestrzenie wyszukiwania i jest bardzo trudny do rozwiązania.

Definicja TSP to: mając dane N miast, jeśli sprzedawca zaczynając od swojego
rodzinnego miasta ma odwiedzić każde miasto dokładnie raz a następnie wrócić do
domu, znajdź taką kolejność zwiedzania, że całkowita przebyta odległość (koszt)
jest minimalna.\cite{IACSIT, Jishou}

Mamy do czynienia z tym problemem na co dzień, chociażby robiąc zakupy, chcemy
przejść jak najmniej i jednocześnie zdobyć wszystkie potrzebne nam artykuły.
Oczywiście problem ten pojawia się w poważniejszej formie w przemyśle,
gdy kurier chce odwiedzić wszystkich klientów w jak najkrótszym czasie, czy
wiertło maszyny CNC ma spędzić jak największą część czasu pracując, nie czekać
na transport na swoje miejsce.

\section{Implementacja}
Do zaimplementowania algorytmu mrówkowego została wykorzystana macierz
sąsiedztwa, w której wagi krawędzi oznaczają odległości między miastami. Każda
mrówka jest początkowo umieszczona na jednym z losowo wybranych miast i w każdym
kolejnym kroku iteracji odwiedza kolejne miasto do momentu odwiedzenia
wszystkich. Pozwala to na szybkie policzenie długości danej trasy.

\subsection*{Opis działania algorytmu}
Poza macierzą sąsiedztwa istnieje również macierz intensywności feromonów.  Ta
macierz jest głównym elementem algorytmu, mrówki w każdej iteracji chodzą po
grafie wybierając kolejne wierzchołki na podstawie macierzy feromonów, jak i
macierzy sąsiedztwa.

Po zakończonej iteracji macierz feromonów jest aktualizowana o informacje
zdobyte przez mrówki.

Na wysokim poziome, wykonywany algorytm jest następujący:
\begin{verbatim}
    while(!shouldTerminate()) {
        simulateAnts();     // na podstawie macierzy incydencji i feromonów
        localSearch();      // opcjonalne, ale bardzo pomocne
        updatePheromones();
    }
\end{verbatim}

Feromony zostawia tylko połowa mrówek o najlepszych (najmniejszych) wynikach.
Oznacza to, że krawędzie z feromonami należą do najlepszych aktualnie
znalezionych ścieżek. Krawędzie z feromonami są częściej wybierane w kolejnych
iteracjach, co daje lepszy początek kolejnym próbą.

Prawdopodobieństwo wybrania krawędzi $ij$ jest podane wzorem
(\ref{eq:probability}):
\begin{align}
    p_{ij} = \frac{\tau_{ij}^\alpha \ \eta_{ij} ^\beta}{\sum_{l\in\mathcal{N}_i} \tau_{il}^\alpha \ \eta_{il} ^\beta} \label{eq:probability}
\end{align}
Gdzie
\begin{itemize}
    \item $\tau_{ij} = m/C$
          \subitem $m$ jest liczbą mrówek
          \subitem $C$ jest długością cyklu
    \item $\eta_{ij} = 1/d_{ij}$
          \subitem $d_{ij}$ jest odległością między wierzchołkami
    \item $\alpha$ i $\beta$ są stałymi, które należy dobrać
\end{itemize}

Powyższy prosty algorytm, zazwyczaj nie dawał zadowalających rezultatów, mimo
dobrania wykładników według literatury. \cite{MIT}

Algorytm został rozszerzony o poniższe ulepszenia:
\paragraph{EAS (Elitist Ant System)}
Gdzie feromony najlepszej mrówki liczą się kilkakrotnie.
\paragraph{AS\textsubscript{rank} (Ranked-based Ant System).}
Gdzie waga feromonów danej mrówki jest zależna od jej miejsca w rankingu
wyników.

Po powyższych ulepszeniach algorytm radził sobie lepiej, niestety nadal nie
optymalnie. Wyniki były często dwukrotnie wyższe od optimum.

Została dodana ostatnia poprawka, mianowicie wyszukiwanie lokalne. Polegające na
analizowaniu cyklu wygenerowanego przez mrówkę i ocenie czy zamiana kolejności
łączenia kolejnych 3 wierzchołków poprawi wynik. Zauważone poprawki są stosowane
przed obliczeniem wyniku dla mrówki. Można rozwiązywać problemy TSP samym tym
algorytmem, niestety szybko staje się to bardzo powolne, gdyż złożoność takiego
sposobu to $O(n^3)$.

Rozwiązania testowane były na nieskierowanych problemach z
biblioteki problemów komiwojażera {\tt TSPLIB}.


\subsection{Konkluzja}
Algorytm mrówkowy wraz z wyszukiwaniem lokalnym daje zazwyczaj wyniki bardzo
bliskie optymalnym, będąc lepszy niż każdy z algorytmów solo. W 89\% przypadków
wyniki były mniej niż 10\% od optimum.

% \bibliographystyle{IEEEtran}
\printbibliography

\end{document}
